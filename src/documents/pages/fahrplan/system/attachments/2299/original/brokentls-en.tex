\documentclass{beamer}
\usepackage{beamerthemesplit}
\usepackage[utf8]{inputenc}
\usepackage{svg}
\usepackage{url}
\begin{document}
\title{How broken is TLS?}
\subtitle{A tale of BEAST, CRIME, Lucky Thirteen and Heartbleed}
\author{Hanno Böck, \url{https://hboeck.de}}
\date{2014-04-19}

\setbeamertemplate{footline}[frame number]

\logo{
\includegraphics[width=1cm]{cc0.png}
}


\titlegraphic{
\includegraphics[height=2cm]{heartbleed.png}
}
\frame{\titlepage
} 


\section{Introduction}

\subsection{Introduction}
\frame{
\begin{itemize}
\item Hanno Böck, freelance journalist, lives in Berlin
\item Often writes about cryptography for Golem.de and others
\item Also runs webhosting (\url{https://schokokeks.org})
\end{itemize}
}

\subsection{Motivation}
\frame{
\begin{itemize}
\item TLS is \textbf{the} most important crypto protocol
\item There are problems on multiple layers (certificate authorities, software, cryptography, protocols)
\item BEAST, CRIME, Lucky Thirteen, Heartbleed, ...
\end{itemize}
}

\subsection{History of SSL / TLS}
\frame{
\begin{itemize}
\item SSL 1 - only internal (Netscape)
\item 1994: SSL 2 (Netscape, severe security issues, disabled)
\item 1996: SSL 3 (Netscape, no extensions, still used)
\item 1999: TLS 1 (IETF standard, problems with CBC, til today de facto standard)
\item 2006: TLS 1.1 (half fix for CBC-problems)
\item 2008: TLS 1.2 (introduces authenticated encryption with GCM and SHA-2)
\end{itemize}
}

\subsection{Overview}

\frame{
\begin{itemize}
\item X.509 certificate authority signs host-certificate, host-certificate used to get a session key for TLS
\item Two protocols (X.509 and TLS)
\item So we have software, CAs, X.509 and TLS itself
\end{itemize}
}


\section{Software}

\subsection{Heartbleed}

\frame{
\begin{itemize}
\item Heartbleed is only a memory read error, just a software bug, nothing with cryptography
\item So pretty boring, right?
\end{itemize}
}

\subsection{Software}

\frame{
\begin{itemize}
\item SSL libraries have terrible code quality
\item OpenSSL Valhalla Rampage - have fun and be scared
\item Don't blame OpenSSL. The others are just as bad
\end{itemize}
}


\subsection{Featurebloat}

\frame{
\begin{itemize}
\item Who needs the heartbeat extension?
\item What other extensions are there in TLS and who needs them?
\item And who needs GOST, DSA, SEED, IDEA, Brainpool curves or Camellia?
(GOST is only a draft, OpenSSL still has code for it)
\item TLS also supports DES or RC2 with 40 Bit
\end{itemize}
}

\subsection{C}

\frame{
\begin{itemize}
\item It is extremely difficult to write secure code in C (buffer overflows etc.)
\item Which programming language is better? Not too exotic, not too much overhead and
usable on many different platforms, but still memsafe?
\item And by the way: Constant time implementations needed
\item Can we do memsafe C? (Softbount/CETS, OpenBSD malloc)
\end{itemize}
}


\section{CAs}

\subsection{Certificate Authorities}
\frame{
\begin{itemize}
\item Main problem: Every certificate authority can do Man-in-the-Middle-attacks on every website
\item You automatically trust all CAs in your browser and all sub CAs
\item Weird: It doesn't matter if your CA is trustworthy, only the least trustworthy CA matters
\end{itemize}
}

\subsection{2011 CA disaster}

\frame{
\begin{itemize}
\item CA-disaster 2011
\item Diginotar, Comodo, Türktrust and others
\item EFF SSL Observatory finds many valid certificates that shouldn't exist (e.g. 512 Bit EV certificate)
\item Diginotar is the only case where it had consequences
\item Comodo issued fake certificates for mail.google.com, www.google.com, login.yahoo.com,
login.skype.com, addons.mozilla.org and login.live.com erstellt
\end{itemize}
}

\subsection{Revocation}
\frame{
\begin{itemize}
\item CRL doesn't scale, OCSP not privacy friendly
\item OCSP useless in Firefox and IE, disabled in Chrome
\item OCSP Stapling could fix things a bit
\item OCSP Stapling Required - only a draft
\item Revocation does not work right now
\end{itemize}
}

\subsection{Revocation that costs money}
\frame{
\begin{itemize}
\item Heartbleed: StartSSL charges for revocation
\item This is a problem: It gives incentives to do the wrong thing
\item Cynics might say: Doesn't matter, it's broken anyway
\item More general problem: It's expensive and difficult to be secure (certificates) - it should be the other way round
\end{itemize}
}


\subsection{Fixing CAs}
\frame{
\begin{itemize}
\item Convergence: distributed access (MitM still possible if attacker near server)
\item Sovereign Keys: EFF, complicated, uses append-only log
\item TACK: draft for TLS to pin certificates, shares many ideas with Sovereign Keys
\item HTTP Key Pinning: draft from Google, only HTTPS
\item Certificate Transparency: from Google, append-only log, make sure MitM gets detected
\end{itemize}
}

\section{X.509}

\subsection{Algorithms}
\frame{
\begin{itemize}
\item Three public key types for X.509 certificates: RSA, DSA, ECDSA
\item Everyone uses RSA and that's a good thing
\item DSA/ECDSA have severe problems with bad randomness
\end{itemize}
}

\subsection{Signaturen}
\frame{
\begin{itemize}
\item MD5 signatures were used until someone showed they are really broken
\item SHA1 signatures are de facto standard
\item Few CAs do SHA256 signatures (about to change)
\item RSA signatures use old PKCS \#1 1.5 scheme, RSA-PSS (PKCS \#1 2.1) would be better, lacks software support
\end{itemize}
}

\section{Symmetric encryption}
\subsection{Algorithms}
\frame{
\begin{itemize}
\item TLS supports a lot of algorithm combinations
\item Example: ECDHE-RSA-AES256-GCM-SHA384
\item Key exchange with elliptic curves, signature with RSA, encryption with AES, key size 256 bit, block mode GCM (Galois/Counter Mode),
MAC algorithm SHA385
\end{itemize}
}

\subsection{BEAST}
\frame{
\begin{itemize}
\item Until TLS 1.0 AES CBC used an implicit initialization vector
\item This led to the BEAST attack
\item Fixed in TLS 1.1, but who implements better protocols with more security if they don't have to?
\end{itemize}
}

\subsection{CBC, MAC, Padding}
\frame{
\begin{itemize}
\item TLS needs confidentiality and authenticity
\item Most common: Encryption with AES-CBC, authentication with HMAC
\item TLS does MAC-then-Pad-then-Encrypt
\item Different error messages for padding and MAC errors allow Padding Oracle attack
\item "Solution": just one error message
\end{itemize}
}

\subsection{From the TLS 1.2 standard}
\frame{
"Canvel et al. [CBCTIME] have demonstrated a
   timing attack on CBC padding based on the time required to compute
   the MAC.  In order to defend against this attack, implementations
   MUST ensure that record processing time is essentially the same
   whether or not the padding is correct.  In general, the best way to
   do this is to compute the MAC even if the padding is incorrect, and
   only then reject the packet.  For instance, if the pad appears to be
   incorrect, the implementation might assume a zero-length pad and then
   compute the MAC.  \textbf{This leaves a small timing channel}, since MAC
   performance depends to some extent on the size of the data fragment,
   \textbf{but it is not believed to be large enough to be exploitable}, due to
   the large block size of existing MACs and the small size of the
   timing signal." (TLS 1.2, RFC 5246, p. 23)
}

\subsection{Lucky Thirteen}
\frame{
\begin{itemize}
\item To translate this: We know there is a problem, but we don't think it's a real problem
\item That was the Lucky Thirteen attack - it was already mentioned in the TLS standard itself
\end{itemize}
}

\subsection{CBC}
\frame{
\begin{itemize}
\item For all problems with BEAST and Lucky Thirteen there are workarounds in browsers
\item But after Lucky Thirteen many wanted to avoid CBC
\item Solution: RC4 (only non-CBC-algorithm left before TLS 1.2)
\item PCI verification (credit card standard) required RC4 for some time
\end{itemize}
}

\subsection{RC4}
\frame{
\begin{itemize}
\item RC4 was developed 1994 by Ron Rivest, not for public use
\item Source was published in a newsgroup, is only a few lines long and you can copy-and-paste it from Wikipedia
\item March 2013: Bernstein et al show an impractical attack on RC4 in TLS, no workarounds
\item After Snowden some people speculate that NSA can attack RC4 in real time
\item Today many people think RC4 has to die (draft)
\end{itemize}
}

\subsection{RC4 or CBC}
\frame{
\begin{itemize}
\item RC4 or CBC with MAC-then-Encrypt? Both are bad, but for CBC we have workarounds
\item BEAST, Lucky Thirteen, RC4-attacks are all not very practical, but this is not good
\item AES-GCM: Authenticated encryption, only in TLS 1.2, only new browsers
\item Encrypt-then-MAC for CBC: draft
\item ChaCha20/Poly1394: draft - very fast and probably very secure cipher
\end{itemize}
}

\section{TLS misc}

\subsection{Forward Secrecy}
\frame{
\begin{itemize}
\item Forward secrecy is great! If you do it right
\item but...
\item Apache before 2.4.7 only supports Diffie-Hellman (DHE) with 1024 bit, Java until 1.7 the same
\item And there are elliptic curves... (ECDHE)
\end{itemize}
}

\subsection{Elliptic Curves}
\frame{
\begin{itemize}
\item Idea: Diffie-Hellman and El Gamal/DSA use a "group" (some math structure)
\item You can use any group you like
\item Assumption: If you use an elliptic curve that's much more secure (the best
known attacks on discrete logarithms don't work in elliptic curves)
\item There are infinitely many elliptic curves - which one should we use?
\end{itemize}
}

\subsection{NIST Curves}
\frame{
\begin{itemize}
\item Solution: We ask the NSA - they have lots of skilled cryptographers, they should know
\item In 1999 NIST published some good, well tested elliptic curves
\item The author works for the NSA
\item Where does 3045ae6f c8422f64 ed579528 d38120ea e12196d5 come from?
\item It is not very likely that the NSA has a backdoor, but we can't completely rule it out
\end{itemize}
}

\subsection{Curve25519}
\frame{
\begin{itemize}
\item Dan Bernstein developed a better elliptic curve: Curve25519, 255 bit
\item Draft for TLS
\item The SafeCurves project also has longer curves
\item ECDSA doesn't work and is bad anyway, there is Ed25519 for Curve25519
\end{itemize}
}

\subsection{Compression}
\frame{
\begin{itemize}
\item Compression leaks information about content and leads to side channel attacks (CRIME attack)
\item Just disable it, your browser won't use it anyway
\item Problem: Similar attacks work on HTTP compression, workarounds are tricky and have to happen in the
web application (BREACH attack)
\end{itemize}
}

\section{Misc}

\subsection{Downgrades}
\frame{
\begin{itemize}
\item TLS has an internal downgrade protection
\item But browsers have a "workaround" for this protection: Try a lower protocol if the higher one fails
\item Problem: Broken middleware and load balancers are everywhere
\item Downgrades can happen with bad connections (mobile Internet)
\item This is a problem for SNI (ServerNameIndication) and other extensions - please disable SSLv3 everywhere
\end{itemize}
}

\subsection{F5 / BIG-IP}
\frame{
\begin{itemize}
\item BIG-IP from F5 is such a story...
\item One of the reasons for those "workarounds"
\item Also there are devices that don't like TLS handshakes between 256 and 512 bytes, so we now
have a TLS padding extension to add some useless data
\item But there is other hardware that breaks with the padding extensions (Ironware SMTP servers from Cisco)
\item Browsers do workarounds for broken stuff all the time
\item There's now a draft for better downgrade protection (a workaround for a workaround for broken stuff)
\end{itemize}
}

\subsection{Frankencerts}
\frame{
\begin{itemize}
\item Frankencerts - take a bunch of real certificates, change things randomly
\item And then check if they are valid
\item In a perfect world every library should come to the same conclusion
\item In the real world they don't
\item This uncovered severe bugs in MatrixSSL and GnuTLS and smaller bugs in various
other TLS implementations
\end{itemize}
}

\subsection{Dual EC DRBG}
\frame{
\begin{itemize}
\item Dual EC DRBG has almost certainly a backdoor from the NSA
\item Research project: We replace the backdoor parameters with our own parameters
\item This revealed private keys for DSA and ECDSA in TLS (told you so: don't use them)
\item RSA BSAFE used Dual EC by default
\item The TLS extended random extension makes this attack easier, there exists a draft
and code in RSA BSAFE (learn: don't invent extensions without a clear purpose. Oh, we already
knew that from Heartbleed)
\item Dual EC rarely used, so no real problem, but exciting research
\end{itemize}
}

\subsection{Triple Handshake}
\frame{
\begin{itemize}
\item Bug in the logic of TLS handshakes and renegotiations
\item Browser workaround, Adam Langley has a check for your browser
\item Interesting fact: some browsers accept insecure parameters for Diffie-Hellman
\item Diffie-Hellman needs a \textbf{large} prime - some browsers think 17 or 15 are large primes
\end{itemize}
}

\subsection{SSL Stripping / HSTS}
\frame{
\begin{itemize}
\item SSL Stripping: Attacker prevents HTTPS by intercepting links or first HTTP access
\item HSTS header forces browser to use TLS and forbids HTTP to same host
\item Some browsers have build-in HSTS domains
\item WARNING: There is no way to do hybrid HTTPS/HTTP solutions secure. Just protecting
the login is wrong
\end{itemize}
}

\subsection{ASN.1}
\frame{
\begin{itemize}
\item ASN.1 is the most horrible binary format to parse
\item Know any good software to handle it? I don't
\end{itemize}
-----BEGIN CERTIFICATE-----
MIICPDCCAaUCED9pHoGc8JpK83P/uUii5N0w DQYJKoZIhvcNAQEFBQAwXzELMAkG
}


\subsection{Quantum Computers}
\frame{
\begin{itemize}
\item Quantum computers kill all public key and key exchange algorithms in TLS
\item 2012 Nobel price for research that could help building quantum computers
\item Post quantum cryptography is very experimental, no algorithm that could be deployed easily
\item Quantum cryptography won't safe you, it is a nice thought experiment, but it is completely impractical
\end{itemize}
}

\subsection{Tipps for server admins}
\frame{
\begin{itemize}
\item Use latest Apache
\item Support TLS 1.2 with AES-GCM and forward secrecy
\item Disable RC4, TLS compression, SSLv3 and all mostly unused algorithms (EXPORT, NULL, SEED, IDEA, ...)
\item Enable OCSP stapling
\item Use RSA certificates with 2048 or 4096 bit, signed with SHA256
\item Use the Qualys SSL test
\end{itemize}
}

\section{End}

\subsection{Two Possible Conclusions}
\frame{
\begin{itemize}
\item a) TLS is broken beyond repair, we need to get rid of it and make something better
\item b) TLS is broken on many layers, but we have ideas and fixes for many problems, we need to get them deployed
\item Choose your own conclusion - and help making it happen
\end{itemize}
}


\subsection{Read and Learn}
\frame{
\begin{itemize}
\item I often write about crypto on Golem.de (German) - \url{http://hboeck.de}
\item Matthew Green`s Blog \url{http://blog.cryptographyengineering.com/}
\item Adam Langley`s Blog \url{https://www.imperialviolet.org/}
\item Dan Bernstein`s Blog (DJB) \url{http://blog.cr.yp.to/}
\item Very good online course by Dan Boneh on Coursera \url{https://www.coursera.org/course/crypto}
\end{itemize}
}

\subsection{Sources}
\frame{
\begin{itemize}
\item TLS 1.2, RFC 5246 \url{http://www.ietf.org/rfc/rfc5246.txt}
\item Heartbleed \url{http://heartbleed.com/}
\item OpenSSL Valhalla Rampage \url{http://opensslrampage.org/}
\item EFF SSL Observatory \url{https://www.eff.org/observatory}
\item Internet scans \url{https://scans.io/}
\item Sovereign Keys \url{https://www.eff.org/sovereign-keys}
\item Convergence \url{http://convergence.io/}
\item TACK \url{http://tack.io/}
\item HTTP Key Pinning \url{http://tools.ietf.org/html/draft-ietf-websec-key-pinning-01}
\item Certificate Transparency \url{http://certificate-transparency.org}
\end{itemize}
}

\subsection{Sources}
\frame{
\begin{itemize}
\item OCSP Stapling Required \url{http://tools.ietf.org/html/draft-hallambaker-tlssecuritypolicy-03}
\item MD5 broken \url{http://media.ccc.de/browse/congress/2008/25c3-3023-en-making\_the\_theoretical\_possible.html}
\item RSA-PSS thesis \url{https://rsapss.hboeck.de/}
\item Padding Oracle \url{http://www.iacr.org/cryptodb/archive/2002/EUROCRYPT/2850/2850.pdf}
\item BEAST \url{http://ekoparty.org/2011/juliano-rizzo.php}
\item Lucky Thirteen \url{http://www.isg.rhul.ac.uk/tls/Lucky13.html}
\item RC4 attack \url{http://www.isg.rhul.ac.uk/tls/}
\item Prohibiting RC4 draft \url{https://tools.ietf.org/html/draft-popov-tls-prohibiting-rc4-02}
\item BREACH \url{http://breachattack.com/}
\end{itemize}
}

\subsection{Sources}
\frame{
\begin{itemize}
\item SafeCurves \url{http://safecurves.cr.yp.to/}
\item Frankencerts \url{https://www.cs.utexas.edu/\textasciitilde shmat/shmat\_oak14.pdf}
\item Dual EC \url{http://dualec.org/} \url{https://projectbullrun.org/dual-ec/}
\item Triple Handshake \url{https://secure-resumption.com/}
\item Browser check for Triple Handshake \url{https://www.imperialviolet.org/2014/03/03/triplehandshake.html}
\item Browser check for DH parameters \url{https://dh.tlsfun.de/}
\item Post quantum cryptography \url{http://pqcrypto.org/}
\item TLS – Nuke it from Orbit \url{http://clearcryptocode.org/tls/}
\item Qualys SSL-Test \url{https://www.ssllabs.com/ssltest/}

\end{itemize}
}

\subsection{Final words}
\frame{
\begin{itemize}
\item Thank you Edward Snowden!
\end{itemize}
\begin{center}
\includegraphics[width=5cm]{snowden.jpg}
\end{center}
}



\end{document}

